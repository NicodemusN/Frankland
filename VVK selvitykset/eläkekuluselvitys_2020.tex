\documentclass{article}
\usepackage[utf8]{inputenc}

\title{Eläkekuluselvitys}
\author{Valtiovarainkanslia}
\date{20.8.2020}

\setcounter{secnumdepth}{0}
\usepackage[left=30mm, right=20mm]{geometry}
\setlength{\parindent}{0pt}
\usepackage{tabularx}

\begin{document}
	
	\maketitle
	
	\newpage
	\section{Eläkekulut vuonna 2021 vuoden 2020 mallilla}
	
	Valtiovarainkanslia on laskenut kulut eläkkeistä vuoden 2021 talousarvioon käyttämällä vuoden 2020 mallia. Mallia tullaan muokkaamaan tarvittaessa, jos valtiontalouden tasapainotus sitä vaatii.
	
	\vspace{12pt}
	
	\begin{table}[htbp]
	\centering
	\begin{tabularx}{\textwidth}{|X|X|X|X|}
		\hline
		Keskimääräiset työelämän vuositulot (€) & Saajien määrä & Tuloluokan kokonaistulot työelämässä (€) & Maksettava eläke (€)\\
		\hline
		6 120 & 398 & 2 437 508 & 1 316 254\\
		\hline
		11 520 & 506 & 5 823 547 & 3 144 716\\
		\hline
		14 880 & 375 & 5 584 576 & 3 015 617\\
		\hline
		19 200 & 164 & 3 147 069 & 1 699 417\\
		\hline
		25 607 & 89 & 2 275 142 & 1 228 577\\
		\hline
		74 773 & 44 & 3 321 708 & 1 793 722\\
		\hline
		624 000 & 22 & 13 860 337 & 7 484 582\\
		\hline
	\end{tabularx}
	\end{table}

	\vspace{12pt}
	
	Maksettavia eläkkeitä tulisi yhteensä 19 682 939 €.
	
	\vspace{12pt}
	Valtiovarainkanslian mukaan on selvää, että maksettavien eläkkeiden summa on kestämätön valtiontaloudelle, koska se synnyttäisi mahdollisesti jopa lähes viiden miljoonan euron kestävyysvajeen. Tämän vuoksi kanslia esittää eläkekertoimen laskemista.
	
	\vspace{12pt}
	Valtiovarainkanslian esitys vuoden 2021 eläkemenoiksi:
	
	\vspace{12pt}
		\begin{table}[htbp]
		\centering
		\begin{tabularx}{\textwidth}{|X|X|X|X|}
			\hline
			Keskimääräiset työelämän vuositulot (€) & Saajien määrä & Tuloluokan kokonaistulot työelämässä (€) & Maksettava eläke (€)\\
			\hline
			6 120 & 398 & 2 437 508 & 1 048 128\\
			\hline
			11 520 & 506 & 5 823 547 & 2 504 125\\
			\hline
			14 880 & 375 & 5 584 576 & 2 401 368\\
			\hline
			19 200 & 164 & 3 147 069 & 1 353 239\\
			\hline
			25 607 & 89 & 2 275 142 & 978 311\\
			\hline
			74 773 & 44 & 3 321 708 & 1 428 334\\
			\hline
			624 000 & 22 & 13 860 337 & 5 959 945\\
			\hline
		\end{tabularx}
	\end{table}

	\vspace{12pt}
	Tässä mallissa maksettavia eläkkeitä tulisi yhteensä 15 673 451 €. Summa voi nousta varsinaisessa valtion talousarviossa, koska siinä otetaan huomioon uuden väestönlaskennan vaikutus eläkeläisten määrään.
	
	\vspace{12pt}
	Allekirjoitus:wwwwwwwwwwwwwwwww
	\underline{Sean Andersson}
	
	
\end{document}