\documentclass{article}
\usepackage[utf8]{inputenc}

\title{Julkisten alojen työehtosopimus}
\author{Valtakunnansovittelijan toimisto / oikeuskanslia}
\date{30.9.2020}

\setcounter{secnumdepth}{0}
\usepackage[left=30mm, right=20mm]{geometry}
\setlength{\parindent}{0pt}

\begin{document}
	
	\maketitle
	
	\newpage
	
	\section{Sovellettavat työlain osa-alueet}
	
	Tässä laissa sovelletaan Työlakia (1/2015) kaikilta niiltä osin, kuin on tarpeen lain asettamien ehtojen täyttämiseksi. Tämä työehtosopimus (TES) perustuu työlain § 2 säätämiin työntekijän ja työnantajan asemiin. Tämän sopimuksen tarkoituksena on edustaa molempia osapuolia ja saavuttaa yhteisymmärrys heidän välilleen työnteon ehdoista.
	
	\vspace{12pt}
	
	Tässä työehtosopimuksessa sovitaan julkisten alojen palkoista terveydenhuollon ja koululaitoksen julkisia työpaikkoja ja virkoja lukuunottamatta. Työnantajapuolta edustaa työlain § 5 mukaisesti Julkisyhteisöjen Työnantajien Kilta (JUTKA) ja työntekijäpuolta edustaa saman pykälän mukaisesti Julkisten Virkojen Yhdistys (JVY).
	
	\vspace{12pt}
	
	Tämän työehtosopimuksen laatimisen oikeuttaa työlain § 6, jonka mukaan työehtosopimuksissa voidaan alakohtaisesti säätää tarkemmin palkoista kansallisen minimipalkkasopimuksen kehysten rajoissa. Tämä työehtosopimus määrittää myös palkanmaksuajan, ennakonpidätysprosentin kehykset ja eläketiedot.
	
	\section{Osapuolien vaatimukset}
	
	JVY:
	\begin{itemize}
		\item Kaikkien julkisyhteisöjen virkojen minimipalkaksi 15 319 (€).
		\item Kaikille ylemmille virkamiehille vastuulisää vähintään 1000 (€) virkatasoa kohden ja 2\%:n vuosi-indeksikorolla ilman inflaatiosuhteutusta.
		\item Työtuntien rajoitus viiteen päivään viikossa korkeintaan kahdeksan tuntia päivässä ilman ylityövaatimusoikeutta työnantajalta.
	\end{itemize}

	\vspace{12pt}
	JUTKA:
	\begin{itemize}
		\item Tulevien veroalennusten vuoksi ei korotuksia palkkoihin.
		\item Ylityövaatimusoikeuden säilyttävä työnantajalla, koska terveydenhuoltouudistuksen myötä virastot tulevat väliaikaisesti ylikuormitetuiksi vuoden 2021 alussa.
	\end{itemize}
	
	\section{Allekirjoitukset}
	\vspace{12pt}
	Vastaajan allekirjoitus:
	
	\underline{Sean Andersson}
	
	
\end{document}