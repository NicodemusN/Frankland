\documentclass{article}
\usepackage[utf8]{inputenc}

\title{Valtiontalouden muutosselvitys}
\author{Valtiovarainkanslia}
\date{1.9.2020}

\setcounter{secnumdepth}{0}
\usepackage[left=30mm, right=20mm]{geometry}
\setlength{\parindent}{0pt}

\begin{document}
	\maketitle
	
	\newpage
	
	\section{Selvityksen tavoitteet}
	
	\fontsize{12}{14}{
	
	Tämän valtiontalouden muutosselvityksen tavoitteena on kartoittaa sosiaalietuuksien rakenteellisten muutoksien mahdollisuuksia ja mahdollisia hyötyjä tai haittoja. Muutosselvitys on tehty yhteistyössä Uussaaren Hypoteekkiyhdistyksen ja sosiaaliasiainkanslian kanssa.
	
	\vspace{12pt}

	Selvityksen keinoina käytettiin tietoja valtiontalouden sosiaalietuuksien rakenteista ja aikaisempia talousarvioita. Uussaaren Hypoteekkiyhdistys toimitti tiedot asumistuen kohdistumisesta ja käytöstä yleisesti väestössä. Sosiaaliasiainkanslia vahvisti tiedot sosiaalietuuksien summista ja saajista.
	}
	
	\section{Selvityksen löydökset}
	
	\fontsize{12}{14}{
	
	Valtiovarainkanslia teki seuraavia havaintoja tutkiessaan sosiaalietuuksia:
	
	\begin{itemize}
		\item sosiaalietuuksien saamisen perusteissa on epäselvyyksiä
		\item kaikki sosiaalietuudet eivät päädy tarkoituksenmukaiseen käyttöönsä
		\item osa sosiaalietuuksista on epäjohdonmukaisia toisiinsa nähden ja jopa turhia erillisinä etuuksinaan
	\end{itemize}

	Valtiovarainkanslia esittää, että asumistuki lakkautettaisiin kokonaan sen saamisen perusteiden epäselvyyksien johdosta ja siksi, että tuki ei todennäköisesti ole päätynyt sen asianmukaiseen tarkoitukseen. Asumistukea on maksettu palveluviraston ja veroviraston toimesta ja alempiin tuloluokkiin kuuluvien kansalaisten tekemien hakemuksien perusteella, joissa tuen saamisen edellytykset ovat olleet hyvin subjektiivisia ja hakijoiden omien kertomusten varassa. Hyväksytyt hakemukset on vahvistettu valtionkanslian määrärahojen puitteissa, eikä tukien tarpeen mukaisesti.
	
	\vspace{12pt}
	Asumistuen lisäksi toimeentulotuki on selkeästi korjauksen tarpeessa. Tukea on myönnetty automaattisesti kaikille alimpaan valtionkanslian määrittelemään tuloluokkaan kuuluville, jotka ovat toimeentulotukea hakeneet. Tukea on myönnetty valtionkanslian myöntämien määrärahojen puitteissa, realistista tuen tarvetta huomioimatta.
	
	\vspace{12pt}
	Valtionvarainkanslia esittää, että asumistukea lakkautettaessa toimeentulotukea laajennetaan ja sen saamisen edellytyksiä tarkennetaan vastaamaan paremmin pienituloisten tarpeisiin heikentämättä kuitenkaan työllistymisen mahdollisuuksia ja kannustimia. Toimeentulotuen myöntämisessä on otettava huomioon myös ostovoiman indeksin kehitys. Toimeentulotukeen vahvasti liittyvä työttömyystuki tarvinnee myös tarkennusta ja parempaa kohdistusta lyhytaikaisesta ja laittomasta irtisanoutumisesta johtuvavasta työttömyydestä kärsiviä kohtaan.
	
	\vspace{12pt}
	Viimeisenä kehityshankkeena valtiovarainkanslia ehdottaa sosiaalietuuksien kirjaamista erilliseen lakiinsa, koska se on perustuslaissa määritelty kansalaisten oikeus. Lakiin kirjattuina valtionkanslian toimet etuuksien suhteet ovat myös tarkempia ja selkeämmin perusteltavissa ja etuuksien poliittinen haavoittuvuus vähenee.
	
	\vspace{12pt}
	Selvityksen vastaaja:
	
	\underline{Sean Andersson}
	
	

}

\end{document}