\documentclass{article}
\usepackage[utf8]{inputenc}

\title{Liberaalipuolueen puoluekokouksen 2021 pöytäkirja}
\author{Puoluevaltuuston kokous / puoluesihteeri}
\date{15.2.2021}

\setcounter{secnumdepth}{0}
\usepackage[left=30mm, right=20mm]{geometry}
\setlength{\parindent}{0pt}

\begin{document}
	
	\maketitle
	
	\section{Puoluekokouksen asialista}
	
	Istuva puolueen puheenjohtaja Sean Andersson on päättänyt väistyä puheenjohtajan paikalta ja antaa tilaa muille puolueaktiiveille. Uuden puheenjohtajan paikalle ehdokkaiksi ovat ilmoittautuneet Chris Linter, Gunnar Gunnarsson ja Miriam Shewett.
	
	\vspace{12pt}
	Kaikki ehdokkaat ovat puolueen sääntöjen mukaisesti puolueen jäseniä. Äänioikeutettuja ovat kaikki puoluevaltuuston jäsenet, eli puolueen kansanedustajat, puoluesihteeri, puolueen varainhoitaja, puolueen tiedotusyhteyshenkilö ja kaupunkiedustushenkilöt, yhteensä 20 henkeä.
	
	\vspace{12pt}
	Äänet jakautuivat seuraavasti: Miriam Shewett 11 ääntä, Gunnar Gunnarsson 8 ääntä ja Chris Linter 1 ääni.
	
	\vspace*{12pt}
	Puoluevaltuusto on enemmistöäänin päättänyt valita Miriam Shewettin Liberaalipuolueen puheenjohtajaksi. Hän aloittaa tehtävässään heti.
\end{document}